\documentclass{article}

\usepackage[english, russian]{babel}
\usepackage{geometry}
\usepackage{graphicx}
\usepackage{listings}
\usepackage{xcolor}
\usepackage[14pt]{extsizes}
\usepackage{amsmath}
\usepackage{setspace}
\usepackage{multirow}
\usepackage{tocloft}
\usepackage{indentfirst} 
\usepackage{lipsum}
\usepackage{caption}
\usepackage{cmap}
\usepackage[utf8]{inputenc}
\usepackage[T2A]{fontenc}
\usepackage{svg}

\captionsetup[figure]{name={Рисунок},labelsep=endash}
\captionsetup[table]{singlelinecheck=false, labelsep=endash}

\renewcommand{\cftsecleader}{\cftdotfill{\cftdotsep}}
\geometry{pdftex, left = 3cm, right = 1cm	, top = 2cm, bottom = 2cm}
\onehalfspacing

\setlength{\parindent}{1,25cm}
\lstdefinestyle{lang}{
	basicstyle=\footnotesize\ttfamily,
	frame=single,
	tabsize=4,	
	breaklines=true
}

\lstset{
	literate=
	{а}{{\selectfont\char224}}1
	{б}{{\selectfont\char225}}1
	{в}{{\selectfont\char226}}1
	{г}{{\selectfont\char227}}1
	{д}{{\selectfont\char228}}1
	{е}{{\selectfont\char229}}1
	{ё}{{\"e}}1
	{ж}{{\selectfont\char230}}1
	{з}{{\selectfont\char231}}1
	{и}{{\selectfont\char232}}1
	{й}{{\selectfont\char233}}1
	{к}{{\selectfont\char234}}1
	{л}{{\selectfont\char235}}1
	{м}{{\selectfont\char236}}1
	{н}{{\selectfont\char237}}1
	{о}{{\selectfont\char238}}1
	{п}{{\selectfont\char239}}1
	{р}{{\selectfont\char240}}1
	{с}{{\selectfont\char241}}1
	{т}{{\selectfont\char242}}1
	{у}{{\selectfont\char243}}1
	{ф}{{\selectfont\char244}}1
	{х}{{\selectfont\char245}}1
	{ц}{{\selectfont\char246}}1
	{ч}{{\selectfont\char247}}1
	{ш}{{\selectfont\char248}}1
	{щ}{{\selectfont\char249}}1
	{ъ}{{\selectfont\char250}}1
	{ы}{{\selectfont\char251}}1
	{ь}{{\selectfont\char252}}1
	{э}{{\selectfont\char253}}1
	{ю}{{\selectfont\char254}}1
	{я}{{\selectfont\char255}}1
	{А}{{\selectfont\char192}}1
	{Б}{{\selectfont\char193}}1
	{В}{{\selectfont\char194}}1
	{Г}{{\selectfont\char195}}1
	{Д}{{\selectfont\char196}}1
	{Е}{{\selectfont\char197}}1
	{Ё}{{\"E}}1
	{Ж}{{\selectfont\char198}}1
	{З}{{\selectfont\char199}}1
	{И}{{\selectfont\char200}}1
	{Й}{{\selectfont\char201}}1
	{К}{{\selectfont\char202}}1
	{Л}{{\selectfont\char203}}1
	{М}{{\selectfont\char204}}1
	{Н}{{\selectfont\char205}}1
	{О}{{\selectfont\char206}}1
	{П}{{\selectfont\char207}}1
	{Р}{{\selectfont\char208}}1
	{С}{{\selectfont\char209}}1
	{Т}{{\selectfont\char210}}1
	{У}{{\selectfont\char211}}1
	{Ф}{{\selectfont\char212}}1
	{Х}{{\selectfont\char213}}1
	{Ц}{{\selectfont\char214}}1
	{Ч}{{\selectfont\char215}}1
	{Ш}{{\selectfont\char216}}1
	{Щ}{{\selectfont\char217}}1
	{Ъ}{{\selectfont\char218}}1
	{Ы}{{\selectfont\char219}}1
	{Ь}{{\selectfont\char220}}1
	{Э}{{\selectfont\char221}}1
	{Ю}{{\selectfont\char222}}1
	{Я}{{\selectfont\char223}}1
}

\DeclareCaptionLabelSeparator{line}{\ --\ }
\DeclareCaptionFont{white}{\color{white}}
\DeclareCaptionFormat{listing}{\colorbox[cmyk]{0.43,0.35,0.35,0.01}{\parbox{\textwidth}{\hspace{15pt}#1#2#3}}}
\captionsetup[lstlisting]{
	singlelinecheck=false,
	labelsep=line
}

\begin{document}
\begin{titlepage}
	\newgeometry{pdftex, left=2cm, right=2cm, top=2.5cm, bottom=2.5cm}
	\fontsize{12pt}{12pt}\selectfont
	\noindent\begin{tabular}{|c|c|}	\hline
	\noindent\begin{minipage}{0.15\textwidth}
		\includegraphics[width=\linewidth]{tools/logo.png}
	\end{minipage} &
	\noindent\begin{minipage}{0.85\textwidth}\centering
		\textbf{\newline Министерство науки и высшего образования Российской Федерации}\\
		\textbf{Федеральное государственное бюджетное образовательное учреждение высшего образования}\\
		\textbf{«Московский государственный технический университет имени Н.Э.~Баумана}\\
		\textbf{(национальный исследовательский университет)»}\\
		\textbf{(МГТУ им. Н.Э.~Баумана)}
	\end{minipage} \\
	\hline	\end{tabular}\newline\newline\newline
	\noindent ФАКУЛЬТЕТ \underline{«Информатика и системы управления»} \newline\newline
	\noindent КАФЕДРА \underline{«Программное обеспечение ЭВМ и информационные технологии»}\newline\newline\newline\newline\newline\newline

	\noindent\begin{minipage}{1.0\textwidth}\centering
		\Large\textbf{       Практикум №1}
		\end{minipage}
		
	\noindent\begin{minipage}{1.0\textwidth}\centering
		\textbf{\newline}	
		\end{minipage}

	\noindent\begin{minipage}{1.0\textwidth}\centering
		\Large\textbf{по дисциплине «Архитектура ЭВМ»}	
		\end{minipage}
		
	\noindent\begin{minipage}{1.0\textwidth}\centering
		\Large\textbf{\newline\newline\newline\newline}	
		\end{minipage}
	
	\noindent\textbf{Тема} \underline{Разработка и отладка программ в вычислительном комплексе Тераграф}
\newline\newline
	\textbf{Студент} \underline{Тузов Даниил Александрович}\newline\newline
	\textbf{Группа} \underline{ИУ7-52Б}\newline\newline
	\textbf{Преподаватель} \underline{Калитвенцев Максим Павлович}
	
	\begin{center}
		\vfill
		Москва, \the\year ~г.
	\end{center}
	\restoregeometry
	\clearpage
\end{titlepage}

\section{Введение}

Практикум посвящен освоению принципов работы вычислительного комплекса Тераграф и получению практических навыков 
решения задач обработки множеств на основе гетерогенной вычислительной структуры. В ходе практикума необходимо 
ознакомиться с типовой структурой двух взаимодействующих программ: хост-подсистемы и программного ядра $sw_kernel$. 
Для выполнения практикума предоставляется доступ к облачной платформе $devlab.bmstu.ru$ с установленными 
ускорительными картами микропроцессора Леонард Эйлер и настроенными средствами сборки проектов.

\clearpage\section{Индивидуальное задание}

Сформировать в хост-подсистеме и передать в SPE две коллекции. Описание коллекций:

\begin{itemize}
	\item[---] $students: student_id, name, enrollment_status$.
	\item[---] $financial_aid: aid_id, student_id, amount, year$.
\end{itemize}

Все текстовые поля коллекций предварительно индексируются и сохраняются в $std::map$ в хост-подсистеме (например, путем 
автоинкремента индекса). В SPE передаются только индексы.

Определить, получал ли студент Артур Назаров (передается в запросе из хост-подсистемы) финансовую помощь в год его 
зачисления?

Эквивалентный запрос на языке $AQL$:

\begin{lstlisting}[style=lang, caption=Запрос на языке AQL]
FOR student IN students
  FILTER student.name == Артур Назаров
  LET enrollmentYear = DATE_YEAR(student.enrollment_status)
  LET aid = FIRST(
    FOR fa IN financial_aid
      FILTER fa.student_id == student.student_id AND fa.year == enrollmentYear
      RETURN fa
  )
  RETURN {
    name: student.name,
    received_aid_in_enrollment_year: aid != null
  }
\end{lstlisting}

\textbf{Объяснение:}

\begin{enumerate}
	\item Находим студента: \\FILTER student.name == Артур Назаров.
	\item Получаем год зачисления: \\ LET enrollmentYear = DATE\_YEAR(student.enrollment\_status).
	\item Ищем запись о финансовой помощи в год зачисления: 
	\begin{itemize}
		\item[---] FILTER fa.student\_id == student.student\_id \\AND fa.year == enrollmentYear.
		\item[---] Используем FIRST для получения первой записи.
	\end{itemize}
	\item Возвращаем результат:
	\begin{itemize}
		\item[---] Если aid != null, значит студент получал помощь.
		\item[---] RETURN \{ name: …, received\_aid\_in\_enrollment\_year: … \}.
	\end{itemize}
\end{enumerate}


\clearpage\section{Описание структур}

В листинге~\ref{structs} приведены описания структур: структуры студента и финансовых выплат. Поля структуры соответствуют
заданию. 

\begin{lstlisting}[style=lang, label=structs, caption=Описание  структур]
struct students {
    using vertex_t = uint32_t;
    int struct_number;
    constexpr students(int struct_number) : struct_number(struct_number) {}
    STRUCT(key) {
        uint32_t student_id;
    };
    STRUCT(val) {
        uint32_t name_idx;
        uint32_t enrollment_status;
    };
    #ifdef __riscv64__
        DEFINE_DEFAULT_KEYVAL(key, val)
    #endif
};

struct financial_aids {
    using vertex_t = uint32_t;
    int struct_number;
    constexpr financial_aids(int struct_number) : struct_number(struct_number) {}
    STRUCT(key) {
        uint32_t aid_id;
    };
    STRUCT(val) {
        uint16_t student_id; 
        uint16_t amount;
        uint16_t year; 
    };
    #ifdef __riscv64__
        DEFINE_DEFAULT_KEYVAL(key, val)
    #endif
};
\end{lstlisting}

\clearpage\section{Описание хост-подсистемы}

В листинге~\ref{host} приведено описание хост-подсистемы, которая инициализирует $map$ студентов, занимает ядро и 
инициализирует поток сообщений программному ядру с информацией о студентах и финансовых выплатах. Затем формируется
запрос от хост-подсистемы к ядру и по результатам этого запроса печатается ответ.

\begin{lstlisting}[style=lang, label=host, caption=Описание хост-подсистемы]
int main(int argc, char** argv) {
	ofstream log("logger.log"); //поток вывода сообщений
	unsigned long long offs=0ull;
	gpc *gpc64_inst; //указатель на класс gpc
	if (argc<2) {
		log<<"Использование: host_main <путь к файлу rawbinary>"<<endl;
		return -1;
	}
	gpc64_inst = new gpc();
	log<<"Открывается доступ к "<<gpc64_inst->gpc_dev_path<<endl;
	if (gpc64_inst->load_swk(argv[1])==0) {
		log<<"Программное ядро загружено из файла "<<argv[1]<<endl;
	}
	else {
		log<<"Ошибка загрузки sw_kernel файла << argv[1]"<<endl;
		return -1;
	}
    log << "Начало инициализации..." << endl;
	gpc64_inst->start(__event__(update_students));
    log << "Таблица студентов" << endl;
    for (const auto& [name, index] : name_index) {
        uint32_t e_status = (1 - index % 2);
        log << "Добавлен студент: id=" << index << " name=" << name << " status=" << e_status << endl;
        gpc64_inst->mq_send(students::key{.student_id = index});
        gpc64_inst->mq_send(students::val{.name_idx = index - 1, .enrollment_status = e_status});
    }
    gpc64_inst->mq_send(-1ull);
    log << "Инициализирована таблица студентов" << endl;
    gpc64_inst->start(__event__(update_financial_aid));
    log << "Таблица финансовой помощи" << endl;
    for (uint32_t aid_id = 0; aid_id < TEST_AIDS_COUNT; ++aid_id) {
        uint16_t stud_id = aid_id % 10;
        uint16_t amount = (aid_id % 3) + 1;
        uint16_t year = 2024;
        log << "Добавлена информация о финансовой помощи: id=" << aid_id << " stud_id=" << stud_id << " amount=" << amount << " year=" << year << endl;
        gpc64_inst->mq_send(financial_aids::key{.aid_id = aid_id});
        gpc64_inst->mq_send(financial_aids::val{
            .student_id = stud_id,
            .amount = amount,
            .year = year
        });
    }
    gpc64_inst->mq_send(-1ull);
    log << "Инициализирована таблица финансовой помощи" << endl;   
    std::string student_name = "Артур Назаров";
    log << "Начало поиска финансовой помощи Артура Назарова" << endl;
    gpc64_inst->start(__event__(select_financials));
    uint32_t name_idx = name_index[student_name];
    gpc64_inst->mq_send(name_idx);
    uint16_t result = gpc64_inst->mq_receive();
    if (result != -1ull) log << "Студент: " << student_name << " получал финансовую помощь в " << result << " году" << endl;
    else log << "Студент: " << student_name << " не получал финансовую помощь" << endl;
    
    delete gpc64_inst;
    return 0;
}
\end{lstlisting}

\clearpage\section{Описание кода обработчика, функционирующего в ядре}

В листинге~\ref{kernel} приведено описание обработчика событий от хост-подсистемы. Обработчик функционирует на базе
ядра $sw\_kernel$. При получении событий на обновление информации о студентах и выплатах, вызываются функции 
$update\_students$ и $update\_financial\_aid$, которые обновляют информацию о студентах и выплатах. Функция 
$select\_financials$ выполняет поиск льгот для переданного в качетсве параметра идентификатора студента. В случае успеха 
возвращается год соответствующей выплаты.

\begin{lstlisting}[style=lang, label=kernel, caption=Описание кода обработчика]
int main(void) {
    lnh_init();
    for (;;) {
        //Wait for event
        event_source = wait_event();
        switch(event_source) {
            case __event__(update_students) : update_students(); break; 
            case __event__(update_financial_aid) : update_financial_aid(); break;
            case __event__(select_financials) : select_financials(); break;
        }
        set_gpc_state(READY);
    }
}

void update_students() {
    while (1) {
        students::key key = students::key::from_int(mq_receive());
        if (key == -1ull) break;
        students::val val = students::val::from_int(mq_receive());
        STUDENTS.ins_async(key, val);
    }
}

void update_financial_aid() {
    while (1) {
        financial_aids::key key = financial_aids::key::from_int(mq_receive());
        if (key == -1ull) break;
        financial_aids::val val = financial_aids::val::from_int(mq_receive());
        FINANCIAL_AIDS.ins_async(key, val);
    }
}

void select_financials() {
    while (1) {
        uint32_t target_name_idx = mq_receive();
        auto student_iter = STUDENTS.nsm(students::key{.student_id = target_name_idx});
        auto aid_iter = FINANCIAL_AIDS.nsm(financial_aids::key{.aid_id = 0});
        while (aid_iter) {
            if (aid_iter.value().student_id == student_iter.key())
                mq_send(aid_iter.value().year);
            aid_iter = FINANCIAL_AIDS.nsm(aid_iter.key());
        }    
        mq_send(-1ull);
    }
}
\end{lstlisting}

\clearpage\section{Вывод}

В ходе работы была разработана хост-подсистема, а так же обработчик программного ядра, выполняющие индивидуальное 
задание. Программа была протестирована --- все тесты пройдены успешно.

\end{document}